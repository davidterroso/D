\chapter{Results and Discussion}\label{ResultsDiscussion}

This chapter presents the results from the experiments described in the \ref{Methods} Methods chapter. These outcomes are organized in the following sections as done in the Methods chapter: ``\ref{FluidSegmentation} Fluid Segmentation'', ``\ref{IntermediateSliceSynthesis} Intermediate Slice Synthesis'', and ``\ref{FluidVolumeEstimation} Fluid Volume Estimation''. After showing the results from the experiments, the factors influencing them are discussed, while providing visual examples of the models' performances. The results are also compared with other similar approaches in literature, approaching the different methods that lead to different results.

\section{Fluid Segmentation}\label{FluidSegmentation}
In this section, the results from the experiments performed in multi-class fluid segmentation are shown. This includes all the runs made in \ref{Experiment1} Experiment 1 and \ref{Experiment2} Experiment 2. In these sections, the resulting Dice coefficients are displayed in tables. Each value corresponds to the mean Dice coefficient computed across all slices present in the validation OCT volumes. The results are shown for each fluid (IRF, SRF, and PED) both grouped by vendor and across all vendors. There is also a column that contains the Dice coefficient when considering all the fluids as a single binary label. Unless specified otherwise, the Dice coefficient calculated for a fluid considers all the slices and not just those that contain that fluid. The values highlighted in bold are the best values obtained in each column. 
\par
These values are presented for each run, which follows the conditions described in the \ref{FluidSegmentation} Fluid Segmentation section. Every four runs that are completed using a different validation fold but follow the same conditions are grouped in a single row, by calculating their mean (row ``Set''). From the 5-fold split, fold 1 was reserved, while the remaining four (0, 2, 3, and 4) are used in training and validation. The fold selected for validation in each run appears in the table's validation fold (``VF'') column. For example, if the fold in the ``VF'' column is 2, then the folds used in training were 0, 3, and 4.
\par
In the following subsections, images displaying the segmentation performed by the models and the respective GT are shown. In these masks, IRF is represented in red, SRF in green, and PED in blue.

\subsection{Experiment 1}

\subsubsection{Experiment 1.1}

The results from the first experiment performed, Experiment 1.1, are shown in Table \ref{tab:Experiment1.1Results}. In this experiment, for each validation fold, two runs were made, while keeping the same conditions. The only difference is the input data that, as it is extracted randomly  as described in \ref{Methods} Methods, is different for every run. 

\begin{table*}[!ht]
	\caption{Dice scores for every vendor and fluid for the runs done in Experiment 1.1. The conditions were the same for both sets except the extracted patches that are different in every run due to the random process of extracting them.}
	\centering
	\resizebox{\textwidth}{!}{\begin{tabular}{|c|c|ccc|ccc|ccc|c|c|c|c|}
		\hline
		% Headers
		\multirow{2}{*}{\textbf{Runs}} & 
		\multirow{2}{*}{\textbf{VF}} & 
		\multicolumn{3}{c|}{\textbf{Cirrus}} & 
		\multicolumn{3}{c|}{\textbf{Spectralis}} & 
		\multicolumn{3}{c|}{\textbf{Topcon}} & 
		\multicolumn{1}{c|}{\multirow{2}{*}{\textbf{IRF}}} & 
		\multirow{2}{*}{\textbf{SRF}} & 
		\multirow{2}{*}{\textbf{PED}} & 
		\multirow{2}{*}{\textbf{Fluid}} \\ \cline{3-11} & &
		\multicolumn{1}{c}{\textbf{IRF}} & 
		\multicolumn{1}{c}{\textbf{SRF}} & 
		\textbf{\textbf{PED}} & 
		\multicolumn{1}{c}{\textbf{IRF}} & 
		\multicolumn{1}{c}{\textbf{SRF}} & 
		\textbf{PED} & 
		\textbf{IRF} & 
		\textbf{SRF} & 
		\textbf{PED} & 
		\multicolumn{1}{c|}{} & & & \\ 
		
		\hline
		
		\textbf{Run 1} & 2 & \multicolumn{1}{c|}{0.138} & \multicolumn{1}{c|}{0.089} & 0.072 & \multicolumn{1}{c|}{0.255} & \multicolumn{1}{c|}{0.331} & 0.163
		& \multicolumn{1}{c|}{0.259} & \multicolumn{1}{c|}{0.446} & 0.056 & 0.200 & 0.254 & 0.083 & 0.163 \\
		
		\textbf{Run 2} & 3 & \multicolumn{1}{c|}{0.138} & \multicolumn{1}{c|}{0.290} & 0.236 & \multicolumn{1}{c|}{0.264} & \multicolumn{1}{c|}{0.670} & 0.652 & \multicolumn{1}{c|}{0.389} & \multicolumn{1}{c|}{0.532} & 0.284 & 0.252 & 0.445 & 0.327 & 0.303 \\
		
		\textbf{Run 3} & 4 & \multicolumn{1}{c|}{0.209} & \multicolumn{1}{c|}{0.158} & 0.024 & \multicolumn{1}{c|}{0.255} & \multicolumn{1}{c|}{0.451} & 0.310 & \multicolumn{1}{c|}{0.286} & \multicolumn{1}{c|}{0.595} & 0.151 & 0.249 & 0.386 & 0.117 & 0.173 \\ 
		
		\textbf{Run 4} & 0 & \multicolumn{1}{c|}{0.166} & \multicolumn{1}{c|}{0.179} & 0.041 & \multicolumn{1}{c|}{0.307} & \multicolumn{1}{c|}{0.343} & 0.243 & \multicolumn{1}{c|}{0.371} & \multicolumn{1}{c|}{0.376} & 0.052 & 0.266 & 0.280 & 0.081 & 0.165 \\ 
		
		\hline
	
		\textbf{Set 1} & - & \multicolumn{1}{c|}{0.16} & \multicolumn{1}{c|}{0.18} & 0.09 & \multicolumn{1}{c|}{0.27} & \multicolumn{1}{c|}{0.45} & 0.34 & \multicolumn{1}{c|}{0.33} & \multicolumn{1}{c|}{0.49} & 0.14 & 0.24 & 0.34 & 0.15 & 0.20 \\ 
	
		\hline
		\hline
	
		\textbf{Run 5} & 2 & \multicolumn{1}{c|}{0.106} & \multicolumn{1}{c|}{0.152} & 0.085 & \multicolumn{1}{c|}{0.290} & \multicolumn{1}{c|}{0.406} & 0.256 & \multicolumn{1}{c|}{0.433} & \multicolumn{1}{c|}{0.443} & 0.065 & 0.250 & 0.296 & 0.110 & 0.175 \\
		
		\textbf{Run 6} & 3 & \multicolumn{1}{c|}{0.370} & \multicolumn{1}{c|}{\textbf{0.297}} & 0.454 & \multicolumn{1}{c|}{\textbf{0.410}} & \multicolumn{1}{c|}{\textbf{0.735}} & \textbf{0.673} & \multicolumn{1}{c|}{0.205} & \multicolumn{1}{c|}{\textbf{0.821}} & \textbf{0.672} & 0.317 & \textbf{0.566} & 0.572 & 0.299 \\
		
		\textbf{Run 7} & 4 & \multicolumn{1}{c|}{\textbf{0.590}} & \multicolumn{1}{c|}{0.265} & \textbf{0.666} & \multicolumn{1}{c|}{0.296} & \multicolumn{1}{c|}{0.372} & 0.498 & \multicolumn{1}{c|}{\textbf{0.687}} & \multicolumn{1}{c|}{0.683} & 0.555 & \textbf{0.593} & 0.460 & \textbf{0.596} & \textbf{0.420} \\ 
		
		\textbf{Run 8} & 0 & \multicolumn{1}{c|}{0.321} & \multicolumn{1}{c|}{0.259} & 0.067 & \multicolumn{1}{c|}{0.352} & \multicolumn{1}{c|}{0.514} & 0.467 & \multicolumn{1}{c|}{0.410} & \multicolumn{1}{c|}{0.610} & 0.152 & 0.249 & 0.386 & 0.117 & 0.173 \\ 
		
		\hline
		
		\textbf{Set 2} & - & \multicolumn{1}{c|}{\textbf{0.35}} & \multicolumn{1}{c|}{\textbf{0.24}} & \textbf{0.32} & \multicolumn{1}{c|}{\textbf{0.34}} & \multicolumn{1}{c|}{\textbf{0.51}} & \textbf{0.47} & \multicolumn{1}{c|}{\textbf{0.43}} & \multicolumn{1}{c|}{\textbf{0.64}} & \textbf{0.36} & \textbf{0.38} & \textbf{0.44} & \textbf{0.36} & \textbf{0.27} \\ 
	
		\hline

	\end{tabular}}
	\label{tab:Experiment1.1Results}
\end{table*}

By looking at the table, a few conclusions can be drawn. First, the difference in performance between runs using the same training OCT volumes is evident. Despite some values being similar for the same training volumes, this trend becomes evident when looking at the mean values in the rows ``Set'', where a significant difference is noted, especially in the Cirrus vendor and the PED fluid. It is also evident that some VFs are more consistent than others. For example, for different extracted patches, the models evaluated on validation fold 2 and validation fold 0 presented similar results, while those evaluated in validation fold 3 showed significant differences when the input patches were changed.
\par
The second conclusion is that, overall, the models are not performing well, as the Dice results are low for every VF. These values are specially low in Cirrus, but better both in Spectralis, and Topcon. One of the reasons the model performed so poorly is due to the input it was receiving. Most of the extracted patches do not capture the transitions from background to retina and from retina to the choroid due to its small size. These transitions are of great importance in fluid segmentation in OCT scans since these transitions represent, among other concepts, the boundaries of the region where fluid can appear. In case the model does not understand these anatomic boundaries, segmentation can be performed outside the retina, which worsens the Dice coefficient.
\par
Despite the performances not being good in every vendor, it is worse in Cirrus. In the literature, it is common to see worse performances in the images obtained with Cirrus and Topcon due to the larger presence of speckle noise in them. However, the worse performance in Cirrus is this experiment was due to the patches extracted from the volumes obtained with devices. The B-scans in these volumes present a larger vertical resolution than those obtained with Topcon and Spectralis devices. Therefore, when extracting a patch of the same size across all devices, each patch from a Cirrus scan captures a smaller area of the retina. This translates to an harder understanding of the transition between background and retina and between retina and choroid, as shown in Figure \ref{fig:CirrusPatchExample}.

\begin{figure}[!ht]
	\centering
	\includegraphics[width=0.18\linewidth]{figures/CirrusPatchExample.png}
	\caption{Example of a patch extracted from a Cirrus OCT volume used in Experiment 1.1. In this patch, while the background is noticeable due to its darker shade, the choroid is harder to be identified by an observer (or a model) due to the lack of context.}
	\label{fig:CirrusPatchExample}
\end{figure}

In Figure \ref{fig:Experiment11Segmentation} it is shown a segmentation performed by the model trained in Run 1 (see Table \ref{tab:Experiment1.1Results}). In this figure it becomes evident that the model learned which areas can be segmented inside the retina, but does not understand how the handle the regions further away. For example, in the area significantly above the retina the background is labeled as SRF. However, the background region closer to the retina is not so frequently labeled as any fluid, since it appears in the patches input to the model. The same problem occurs with the oversegmentation of PED in the choroid region, which does not appear in the input patches significantly.
\par
Oversegmentation beyond the retinal bounds is not exclusive to the Cirrus volumes, as it also appears in the OCT scans from other vendors. This suggests that the issue is primarily due to the small patch size rather than the smaller retinal area captured in Cirrus patches, despite this amplifying the problem. 
\par
In the same figure, it is seen that the model has not learned the anatomical relationships between fluids, since it segmented IRF and PED close to each other. This occurs because the model has not learned the relationship between the retinal landmarks and the fluid types due to the small sized inputs.

\begin{figure}[!ht]
	\centering
	\includegraphics[width=1.0\linewidth]{figures/Experiment11Segmentation.png}
	\caption{Example of a poor segmentation made by the model trained in Run 1 (right). In the left, the GT mask for the same image is shown.}
	\label{fig:Experiment11Segmentation}
\end{figure}

\subsubsection{Experiment 1.2}

The resulting Dice coefficient values obtained in Experiment 1.2, where patches of shape 496 $\times$ 512 were used, are shown in Table \ref{tab:Experiment1.2Results}.

\begin{table*}[!ht]
	\caption{Dice scores for every vendor and fluid for the runs done in Experiment 1.2.}
	\centering
	\resizebox{\textwidth}{!}{\begin{tabular}{|c|c|ccc|ccc|ccc|c|c|c|c|}
		\hline
		% Headers
		\multirow{2}{*}{\textbf{Runs}} & 
		\multirow{2}{*}{\textbf{VF}} & 
		\multicolumn{3}{c|}{\textbf{Cirrus}} & 
		\multicolumn{3}{c|}{\textbf{Spectralis}} & 
		\multicolumn{3}{c|}{\textbf{Topcon}} & 
		\multicolumn{1}{c|}{\multirow{2}{*}{\textbf{IRF}}} & 
		\multirow{2}{*}{\textbf{SRF}} & 
		\multirow{2}{*}{\textbf{PED}} & 
		\multirow{2}{*}{\textbf{Fluid}} \\ \cline{3-11} & &
		\multicolumn{1}{c}{\textbf{IRF}} & 
		\multicolumn{1}{c}{\textbf{SRF}} & 
		\textbf{\textbf{PED}} & 
		\multicolumn{1}{c}{\textbf{IRF}} & 
		\multicolumn{1}{c}{\textbf{SRF}} & 
		\textbf{PED} & 
		\textbf{IRF} & 
		\textbf{SRF} & 
		\textbf{PED} & 
		\multicolumn{1}{c|}{} & & & \\ 
			
		\hline
		
		\textbf{Run 9} & 2 & \multicolumn{1}{c|}{0.291} & \multicolumn{1}{c|}{0.450} & 0.281 & \multicolumn{1}{c|}{0.472} & \multicolumn{1}{c|}{0.638} & 0.394 & \multicolumn{1}{c|}{0.505} & \multicolumn{1}{c|}{0.647} & 0.573 & 0.396 & 0.551 & 0.400 & 0.393 \\

		
		\textbf{Run 10} & 3 & \multicolumn{1}{c|}{\textbf{0.586}} & \multicolumn{1}{c|}{\textbf{0.619}} & \textbf{0.727} & \multicolumn{1}{c|}{0.482} & \multicolumn{1}{c|}{\textbf{0.780}} & \textbf{0.698} & \multicolumn{1}{c|}{\textbf{0.749}} & \multicolumn{1}{c|}{\textbf{0.793}} & \textbf{0.788} & \textbf{0.627} & \textbf{0.711} & \textbf{0.744} & \textbf{0.667} \\

		
		\textbf{Run 11} & 4 & \multicolumn{1}{c|}{0.281} & \multicolumn{1}{c|}{0.453} & 0.415 & \multicolumn{1}{c|}{0.429} & \multicolumn{1}{c|}{0.503} & 0.322 & \multicolumn{1}{c|}{0.228} & \multicolumn{1}{c|}{0.532} & 0.324 & 0.278 & 0.494 & 0.363 & 0.296 \\
		
		\textbf{Run 12} & 0 & \multicolumn{1}{c|}{0.242} & \multicolumn{1}{c|}{0.334} & 0.336 & \multicolumn{1}{c|}{\textbf{0.551}} & \multicolumn{1}{c|}{0.564} & 0.423 & \multicolumn{1}{c|}{0.321} & \multicolumn{1}{c|}{0.643} & 0.407 & 0.325 & 0.488 & 0.377 & 0.339 \\
		
		\hline
		
		\textbf{Set 3} & - & \multicolumn{1}{c|}{0.35} & \multicolumn{1}{c|}{0.46} & 0.44 & \multicolumn{1}{c|}{0.48} & \multicolumn{1}{c|}{0.62} & 0.46 & \multicolumn{1}{c|}{0.45} & \multicolumn{1}{c|}{0.65} & 0.52 & 0.41 & 0.56 & 0.47 & 0.42 \\
		
		\hline
					
	\end{tabular}}
	\label{tab:Experiment1.2Results}
\end{table*}

The information resumed in this table allows the comparison between the performance in models that used smaller patches in Experiment 1.1 with those that used larger patches in this Experiment.
\par
By comparing the ``Set'' rows, it becomes evident an overall increase in performance when using larger patches. In fact, it is observed an increase in almost all the mean values when compared to the best performing set in Experiment 1.1. This increase is larger than one decimal point in some columns, and the largest improvements occur in the volumes obtained using the Cirrus device and when segmenting SRF or PED.
\par
The improvement in performance in Cirrus devices can be explained with the more contextual information given. In Experiment 1.1, the Cirrus patches input to the model were not big enough for the network to capture and understand the retinal borders or the relationship between them and the fluids, thus leading to oversegmentation beyond the retina, as explained. In Experiment 1.2, as the same patch covers the retinal layer and the background simultaneously, the model has learned to not segment beyond the retina. In Figure \ref{fig:BigPatchesSegmentationCirrus}, the B-scan shown in Figure \ref{fig:Experiment11Segmentation}, is segmented by the model trained in Run 9 and it is possible to see that the labeling of the background as fluid is no longer happening.
\par 
This also partially justifies the significant improvement in the SRF and PED Dice coefficient. By providing inputs with enough context, the model is no longer segmenting these fluids outside the retina. However, in Figure \ref{fig:BigPatchesSegmentationCirrus} it is also seen that the model is no longer confusing the fluids in the retina, as it no longer segments the PED close to the IRF. This further justifies that the model is learning the anatomical references associated with PED, as confirmed in Figure \ref{fig:BigPatchesSegmentationTopcon}.
\par
Lastly, it is important to note that the IRF did not improve as much as the other fluids. This happens because the input patches in Experiment 1.1 where extracted mainly from the retina, providing all the anatomical context needed for the IRF segmentation.
\par
In Figure \ref{fig:BigPatchesSegmentationTopcon}, it is seen that the model confuses the choroid with the retina, as it labels parts of that region as IRF and PED. The likely cause for this confusion is that the input patches are often cropped in the middle of the retinal layer (as illustrated in Figure \ref{fig:BigPatchExtraction}), which leads to the model not understanding the position of the choroid relative to the retina. This inaccurate IRF and PED segmentation is probably based on the visual resemblance between the structures seen in the choroid and the fluid pockets in the retina. This indicates that the model does not know the location of the choroid.

\begin{figure}[!ht]
	\centering
	\includegraphics[width=0.5\linewidth]{figures/BigPatchSegmentationCirrus.png}
	\caption{Example of the segmentation done by the model trained in Run 9 (right) and its respective GT mask (left). The B-scan segmented is the same as in Figure \ref{fig:Experiment11Segmentation}.}
	\label{fig:BigPatchesSegmentationCirrus}
\end{figure}

\begin{figure}[!ht]
	\centering
	\includegraphics[width=0.6\linewidth]{figures/BigPatchSegmentationTopcon.png}
	\caption{Example of the segmentation done by the model trained in Run 12 (right) and its respective GT mask (left). It is noticeable that the model confuses the choroid with the retina, as segmentation of IRF and PED is performed in the choroid.}
	\label{fig:BigPatchesSegmentationTopcon}
\end{figure}

\subsubsection{Experiment 1.3}
Experiment 1.3 contains all the runs that were performed using vertical patches as input of the multi-class segmentation U-Net. In the first two sets, shown in Table \ref{tab:Experiment1.3FourPatches}, the models were trained using four vertical patches, obtained as explained in the \ref{Methods} Methods chapter.

\begin{table*}[!ht]
	\caption{Dice scores for every vendor and fluid for the runs done in Experiment 1.3 using four vertical patches extracted from each B-scan. In ``Set 4'', the models were trained in 100 epochs, while in ``Set 5'' the models were trained in 200 epochs.}
	\centering
	\resizebox{\textwidth}{!}{\begin{tabular}{|c|c|ccc|ccc|ccc|c|c|c|c|}
		\hline
		% Headers
		\multirow{2}{*}{\textbf{Runs}} & 
		\multirow{2}{*}{\textbf{VF}} & 
		\multicolumn{3}{c|}{\textbf{Cirrus}} & 
		\multicolumn{3}{c|}{\textbf{Spectralis}} & 
		\multicolumn{3}{c|}{\textbf{Topcon}} & 
		\multicolumn{1}{c|}{\multirow{2}{*}{\textbf{IRF}}} & 
		\multirow{2}{*}{\textbf{SRF}} & 
		\multirow{2}{*}{\textbf{PED}} & 
		\multirow{2}{*}{\textbf{Fluid}} \\ \cline{3-11} & &
		\multicolumn{1}{c}{\textbf{IRF}} & 
		\multicolumn{1}{c}{\textbf{SRF}} & 
		\textbf{\textbf{PED}} & 
		\multicolumn{1}{c}{\textbf{IRF}} & 
		\multicolumn{1}{c}{\textbf{SRF}} & 
		\textbf{PED} & 
		\textbf{IRF} & 
		\textbf{SRF} & 
		\textbf{PED} & 
		\multicolumn{1}{c|}{} & & & \\ 
			
		\hline
		
		\textbf{Run 13} & 2 & \multicolumn{1}{c|}{0.411} & \multicolumn{1}{c|}{0.665} & 0.498 & \multicolumn{1}{c|}{0.654} & \multicolumn{1}{c|}{0.735} & 0.611 & \multicolumn{1}{c|}{0.731} & \multicolumn{1}{c|}{0.743} & 0.519 & 0.563 & 0.704 & 0.526 & 0.581 \\

		\textbf{Run 14} & 3 & \multicolumn{1}{c|}{0.390} & \multicolumn{1}{c|}{0.520} & 0.217 & \multicolumn{1}{c|}{0.403} & \multicolumn{1}{c|}{0.564} & 0.429 & \multicolumn{1}{c|}{0.378} & \multicolumn{1}{c|}{0.754} & 0.477 & 0.388 & 0.614 & 0.350 & 0.386 \\
		
		\textbf{Run 15} & 4 & \multicolumn{1}{c|}{0.792} & \multicolumn{1}{c|}{\textbf{0.846}} & \textbf{0.883} & \multicolumn{1}{c|}{0.732} & \multicolumn{1}{c|}{\textbf{0.901}} & 0.733 & \multicolumn{1}{c|}{0.615} & \multicolumn{1}{c|}{0.758} & 0.656 & 0.707 & 0.815 & 0.765 & 0.652 \\
		
		\textbf{Run 16} & 0 & \multicolumn{1}{c|}{\textbf{0.810}} & \multicolumn{1}{c|}{0.834} & 0.715 & \multicolumn{1}{c|}{0.779} & \multicolumn{1}{c|}{0.857} & 0.753 & \multicolumn{1}{c|}{0.783} & \multicolumn{1}{c|}{\textbf{0.924}} & \textbf{0.882} & \textbf{0.795} & \textbf{0.871} & \textbf{0.783} & \textbf{0.709} \\
		
		\hline
		
		\textbf{Set 4} & - & \multicolumn{1}{c|}{\textbf{0.60}} & \multicolumn{1}{c|}{\textbf{0.72}} & 0.58 & \multicolumn{1}{c|}{0.64} & \multicolumn{1}{c|}{\textbf{0.76}} & 0.63 & \multicolumn{1}{c|}{0.63} & \multicolumn{1}{c|}{\textbf{0.79}} & 0.63 & 0.61 & \textbf{0.75} & 0.61 & 0.58 \\
		
		\hline
		\hline
		
		\textbf{Run 17} & 2 & \multicolumn{1}{c|}{0.522} & \multicolumn{1}{c|}{0.784} & 0.703 & \multicolumn{1}{c|}{\textbf{0.805}} & \multicolumn{1}{c|}{0.765} & \textbf{0.865} & \multicolumn{1}{c|}{\textbf{0.828}} & \multicolumn{1}{c|}{0.879} & 0.835 & 0.677 & 0.813 & 0.777 & 0.684 \\
	
		\textbf{Run 18} & 3 & \multicolumn{1}{c|}{0.509} & \multicolumn{1}{c|}{0.654} & 0.598 & \multicolumn{1}{c|}{0.590} & \multicolumn{1}{c|}{0.806} & 0.755 & \multicolumn{1}{c|}{0.777} & \multicolumn{1}{c|}{0.787} & 0.792 & 0.621 & 0.729 & 0.697 & 0.622 \\
					
		\textbf{Run 19} & 4 & \multicolumn{1}{c|}{0.377} & \multicolumn{1}{c|}{0.387} & 0.398 & \multicolumn{1}{c|}{0.478} & \multicolumn{1}{c|}{0.624} & 0.398 & \multicolumn{1}{c|}{0.483} & \multicolumn{1}{c|}{0.729} & 0.448 & 0.436 & 0.567 & 0.420 & 0.399 \\
		
		\textbf{Run 20} & 0 & \multicolumn{1}{c|}{0.753} & \multicolumn{1}{c|}{0.697} & 0.696 & \multicolumn{1}{c|}{0.779} & \multicolumn{1}{c|}{0.844} & 0.783 & \multicolumn{1}{c|}{0.691} & \multicolumn{1}{c|}{0.720} & 0.671 & 0.735 & 0.731 & 0.702 & 0.637 \\
		
		\hline
		
		\textbf{Set 5} & - & \multicolumn{1}{c|}{0.54} & \multicolumn{1}{c|}{0.63} & \textbf{0.60} & \multicolumn{1}{c|}{\textbf{0.66}} & \multicolumn{1}{c|}{0.76} & \textbf{0.70} & \multicolumn{1}{c|}{\textbf{0.69}} & \multicolumn{1}{c|}{0.78} & \textbf{0.69} & \textbf{0.62} & 0.71 & \textbf{0.65} & \textbf{0.59} \\
		
		\hline
			
	\end{tabular}}
	\label{tab:Experiment1.3FourPatches}
\end{table*}

\begin{table*}[!ht]
	\caption{}
	\centering
	\resizebox{\textwidth}{!}{\begin{tabular}{|c|c|ccc|ccc|ccc|c|c|c|c|}
		\hline
		% Headers
		\multirow{2}{*}{\textbf{Runs}} & 
		\multirow{2}{*}{\textbf{VF}} & 
		\multicolumn{3}{c|}{\textbf{Cirrus}} & 
		\multicolumn{3}{c|}{\textbf{Spectralis}} & 
		\multicolumn{3}{c|}{\textbf{Topcon}} & 
		\multicolumn{1}{c|}{\multirow{2}{*}{\textbf{IRF}}} & 
		\multirow{2}{*}{\textbf{SRF}} & 
		\multirow{2}{*}{\textbf{PED}} & 
		\multirow{2}{*}{\textbf{Fluid}} \\ \cline{3-11} & &
		\multicolumn{1}{c}{\textbf{IRF}} & 
		\multicolumn{1}{c}{\textbf{SRF}} & 
		\textbf{\textbf{PED}} & 
		\multicolumn{1}{c}{\textbf{IRF}} & 
		\multicolumn{1}{c}{\textbf{SRF}} & 
		\textbf{PED} & 
		\textbf{IRF} & 
		\textbf{SRF} & 
		\textbf{PED} & 
		\multicolumn{1}{c|}{} & & & \\ 
			
		\hline
		
		\textbf{Run 21} & 2 & \multicolumn{1}{c|}{0.556} & \multicolumn{1}{c|}{0.837} & 0.672 & \multicolumn{1}{c|}{\textbf{0.761}} & \multicolumn{1}{c|}{\textbf{0.853}} & \textbf{0.848} & \multicolumn{1}{c|}{\textbf{0.829}} & \multicolumn{1}{c|}{0.908} & \textbf{0.858} & 0.685 & \textbf{0.864} & 0.767 & \textbf{0.681} \\
		
		\textbf{Run 22} & 3 & \multicolumn{1}{c|}{\textbf{0.734}} & \multicolumn{1}{c|}{\textbf{0.855}} & 0.836 & \multicolumn{1}{c|}{0.636} & \multicolumn{1}{c|}{0.846} & 0.689 & \multicolumn{1}{c|}{0.686} & \multicolumn{1}{c|}{0.781} & 0.731 & \textbf{0.700} & 0.822 & \textbf{0.771} & 0.672 \\
		
		\hline
		\hline
		
		\textbf{Run 23} & 2 & \multicolumn{1}{c|}{0.500} & \multicolumn{1}{c|}{0.762} & 0.635 & \multicolumn{1}{c|}{0.672} & \multicolumn{1}{c|}{0.790} & 0.773 & \multicolumn{1}{c|}{0.819} & \multicolumn{1}{c|}{\textbf{0.936}} & 0.805 & 0.639 & 0.826 & 0.718 & 0.637 \\
		
		\textbf{Run 24} & 3 & \multicolumn{1}{c|}{0.612} & \multicolumn{1}{c|}{0.648} & \textbf{0.882} & \multicolumn{1}{c|}{0.455} & \multicolumn{1}{c|}{0.636} & 0.548 & \multicolumn{1}{c|}{0.499} & \multicolumn{1}{c|}{0.593} & 0.668 & 0.542 & 0.622 & 0.745 & 0.536 \\
		
		\hline
			
	\end{tabular}}
	\label{tab:Experiment1.3SevenVsThirteenPatches}
\end{table*}

\begin{table*}[!ht]
	\caption{}
	\centering
	\resizebox{\textwidth}{!}{\begin{tabular}{|c|c|ccc|ccc|ccc|c|c|c|c|}
			\hline
			% Headers
			\multirow{2}{*}{\textbf{Runs}} & 
			\multirow{2}{*}{\textbf{VF}} & 
			\multicolumn{3}{c|}{\textbf{Cirrus}} & 
			\multicolumn{3}{c|}{\textbf{Spectralis}} & 
			\multicolumn{3}{c|}{\textbf{Topcon}} & 
			\multicolumn{1}{c|}{\multirow{2}{*}{\textbf{IRF}}} & 
			\multirow{2}{*}{\textbf{SRF}} & 
			\multirow{2}{*}{\textbf{PED}} & 
			\multirow{2}{*}{\textbf{Fluid}} \\ \cline{3-11} & &
			\multicolumn{1}{c}{\textbf{IRF}} & 
			\multicolumn{1}{c}{\textbf{SRF}} & 
			\textbf{\textbf{PED}} & 
			\multicolumn{1}{c}{\textbf{IRF}} & 
			\multicolumn{1}{c}{\textbf{SRF}} & 
			\textbf{PED} & 
			\textbf{IRF} & 
			\textbf{SRF} & 
			\textbf{PED} & 
			\multicolumn{1}{c|}{} & & & \\ 
			
			\hline
			
			\textbf{Run 25} & 2 & \multicolumn{1}{c|}{0.406} & \multicolumn{1}{c|}{0.789} & 0.656 & \multicolumn{1}{c|}{0.595} & \multicolumn{1}{c|}{\textbf{0.805}} & \textbf{0.786} & \multicolumn{1}{c|}{0.762} & \multicolumn{1}{c|}{0.841} & 0.797 & 0.560 & 0.809 & 0.727 & 0.620 \\

			
			\textbf{Run 26} & 3 & \multicolumn{1}{c|}{0.589} & \multicolumn{1}{c|}{0.775} & 0.553 & \multicolumn{1}{c|}{0.586} & \multicolumn{1}{c|}{0.803} & 0.766 & \multicolumn{1}{c|}{0.769} & \multicolumn{1}{c|}{\textbf{0.827}} & 0.755 & 0.654 & 0.799 & 0.664 & 0.617 \\
			
			
			\textbf{Run 27} & 4 & \multicolumn{1}{c|}{\textbf{0.797}} & \multicolumn{1}{c|}{\textbf{0.845}} & \textbf{0.827} & \multicolumn{1}{c|}{\textbf{0.734}} & \multicolumn{1}{c|}{0.782} & 0.659 & \multicolumn{1}{c|}{0.533} & \multicolumn{1}{c|}{0.756} & 0.553 & 0.674 & 0.798 & 0.686 & 0.626 \\
			
			
			\textbf{Run 28} & 0 & \multicolumn{1}{c|}{0.677} & \multicolumn{1}{c|}{0.842} & 0.699 & \multicolumn{1}{c|}{0.681} & \multicolumn{1}{c|}{0.799} & 0.732 & \multicolumn{1}{c|}{\textbf{0.791}} & \multicolumn{1}{c|}{0.846} & \textbf{0.854} & \textbf{0.719} & \textbf{0.836} & \textbf{0.762} & \textbf{0.665} \\
			
			
			\hline
			
			\textbf{Set 6} & - & \multicolumn{1}{c|}{0.62} & \multicolumn{1}{c|}{0.81} & 0.68 & \multicolumn{1}{c|}{0.65} & \multicolumn{1}{c|}{0.80} & 0.74 & \multicolumn{1}{c|}{0.71} & \multicolumn{1}{c|}{0.82} & 0.74 & 0.65 & 0.81 & 0.71 & 0.63 \\
			
			\hline
			
	\end{tabular}}
	\label{tab:Experiment1.3SevenPatchesNoRotation}
\end{table*}

\begin{table*}[!ht]
	\caption{}
	\centering
	\resizebox{\textwidth}{!}{\begin{tabular}{|c|c|ccc|ccc|ccc|c|c|c|c|}
		\hline
		% Headers
		\multirow{2}{*}{\textbf{Runs}} & 
		\multirow{2}{*}{\textbf{VF}} & 
		\multicolumn{3}{c|}{\textbf{Cirrus}} & 
		\multicolumn{3}{c|}{\textbf{Spectralis}} & 
		\multicolumn{3}{c|}{\textbf{Topcon}} & 
		\multicolumn{1}{c|}{\multirow{2}{*}{\textbf{IRF}}} & 
		\multirow{2}{*}{\textbf{SRF}} & 
		\multirow{2}{*}{\textbf{PED}} & 
		\multirow{2}{*}{\textbf{Fluid}} \\ \cline{3-11} & &
		\multicolumn{1}{c}{\textbf{IRF}} & 
		\multicolumn{1}{c}{\textbf{SRF}} & 
		\textbf{\textbf{PED}} & 
		\multicolumn{1}{c}{\textbf{IRF}} & 
		\multicolumn{1}{c}{\textbf{SRF}} & 
		\textbf{PED} & 
		\textbf{IRF} & 
		\textbf{SRF} & 
		\textbf{PED} & 
		\multicolumn{1}{c|}{} & & & \\ 
		
		\hline
		
		\textbf{Run 29} & 2 & \multicolumn{1}{c|}{0.497} & \multicolumn{1}{c|}{0.783} & 0.609 & \multicolumn{1}{c|}{0.763} & \multicolumn{1}{c|}{0.851} & 0.822 & \multicolumn{1}{c|}{0.832} & \multicolumn{1}{c|}{\textbf{0.904}} & 0.799 & 0.658 & 0.836 & 0.712 & 0.666 \\
		
		\textbf{Run 30} & 3 & \multicolumn{1}{c|}{0.475} & \multicolumn{1}{c|}{0.747} & 0.613 & \multicolumn{1}{c|}{0.656} & \multicolumn{1}{c|}{0.846} & 0.757 & \multicolumn{1}{c|}{\textbf{0.856}} & \multicolumn{1}{c|}{0.869} & \textbf{0.835} & 0.646 & 0.809 & 0.720 & 0.646 \\
		
		\textbf{Run 31} & 4 & \multicolumn{1}{c|}{0.643} & \multicolumn{1}{c|}{0.709} & 0.747 & \multicolumn{1}{c|}{0.627} & \multicolumn{1}{c|}{0.800} & 0.635 & \multicolumn{1}{c|}{0.457} & \multicolumn{1}{c|}{0.599} & 0.529 & 0.560 & 0.673 & 0.638 & 0.523 \\
		
		\textbf{Run 32} & 0 & \multicolumn{1}{c|}{\textbf{0.794}} & \multicolumn{1}{c|}{\textbf{0.878}} & \textbf{0.770} & \multicolumn{1}{c|}{\textbf{0.807}} & \multicolumn{1}{c|}{\textbf{0.902}} & \textbf{0.862} & \multicolumn{1}{c|}{0.796} & \multicolumn{1}{c|}{0.842} & 0.830 & \textbf{0.797} & \textbf{0.869} & \textbf{0.808} & \textbf{0.698} \\
		
		\hline
		
		\textbf{Set 7} & - & \multicolumn{1}{c|}{0.60} & \multicolumn{1}{c|}{0.78} & 0.68 & \multicolumn{1}{c|}{0.71} & \multicolumn{1}{c|}{0.85} & 0.77 & \multicolumn{1}{c|}{0.74} & \multicolumn{1}{c|}{0.80} & 0.75 & 0.67 & 0.80 & 0.72 & 0.63 \\
		
		\hline
			
	\end{tabular}}
	\label{tab:Experiment1.3SevenPatches5DegreeRotation}
\end{table*}

\begin{table*}[!ht]
	\caption{}
	\centering
	\resizebox{\textwidth}{!}{\begin{tabular}{|c|c|ccc|ccc|ccc|c|c|c|c|}
		\hline
		% Headers
		\multirow{2}{*}{\textbf{Runs}} & 
		\multirow{2}{*}{\textbf{VF}} & 
		\multicolumn{3}{c|}{\textbf{Cirrus}} & 
		\multicolumn{3}{c|}{\textbf{Spectralis}} & 
		\multicolumn{3}{c|}{\textbf{Topcon}} & 
		\multicolumn{1}{c|}{\multirow{2}{*}{\textbf{IRF}}} & 
		\multirow{2}{*}{\textbf{SRF}} & 
		\multirow{2}{*}{\textbf{PED}} & 
		\multirow{2}{*}{\textbf{Fluid}} \\ \cline{3-11} & &
		\multicolumn{1}{c}{\textbf{IRF}} & 
		\multicolumn{1}{c}{\textbf{SRF}} & 
		\textbf{\textbf{PED}} & 
		\multicolumn{1}{c}{\textbf{IRF}} & 
		\multicolumn{1}{c}{\textbf{SRF}} & 
		\textbf{PED} & 
		\textbf{IRF} & 
		\textbf{SRF} & 
		\textbf{PED} & 
		\multicolumn{1}{c|}{} & & & \\ 
			
		\hline
		
		\textbf{Run 33} & 2 & \multicolumn{1}{c|}{0.598} & \multicolumn{1}{c|}{\textbf{0.824}} & 0.719 & \multicolumn{1}{c|}{\textbf{0.805}} & \multicolumn{1}{c|}{0.873} & \textbf{0.887} & \multicolumn{1}{c|}{\textbf{0.848}} & \multicolumn{1}{c|}{\textbf{0.947}} & \textbf{0.863} & \textbf{0.720} & \textbf{0.874} & \textbf{0.798} & \textbf{0.733} \\
		
		\textbf{Run 34} & 3 & \multicolumn{1}{c|}{0.298} & \multicolumn{1}{c|}{0.701} & 0.436 & \multicolumn{1}{c|}{0.442} & \multicolumn{1}{c|}{0.785} & 0.720 & \multicolumn{1}{c|}{0.717} & \multicolumn{1}{c|}{0.814} & 0.746 & 0.477 & 0.757 & 0.599 & 0.530 \\
		
		\textbf{Run 35} & 4 & \multicolumn{1}{c|}{0.670} & \multicolumn{1}{c|}{0.796} & \textbf{0.850} & \multicolumn{1}{c|}{0.531} & \multicolumn{1}{c|}{0.825} & 0.701 & \multicolumn{1}{c|}{0.509} & \multicolumn{1}{c|}{0.703} & 0.674 & 0.582 & 0.759 & 0.754 & 0.578 \\
		
		\textbf{Run 36} & 0 & \multicolumn{1}{c|}{\textbf{0.695}} & \multicolumn{1}{c|}{0.816} & 0.737 & \multicolumn{1}{c|}{0.761} & \multicolumn{1}{c|}{\textbf{0.882}} & 0.838 & \multicolumn{1}{c|}{0.728} & \multicolumn{1}{c|}{0.867} & 0.775 & 0.719 & 0.846 & 0.769 & 0.635 \\
		
		\hline
		
		\textbf{Set 8} & - & \multicolumn{1}{c|}{0.57} & \multicolumn{1}{c|}{0.78} & 0.69 & \multicolumn{1}{c|}{0.63} & \multicolumn{1}{c|}{0.84} & 0.79 & \multicolumn{1}{c|}{0.70} & \multicolumn{1}{c|}{0.83} & 0.76 & 0.62 & 0.81 & 0.73 & 0.62 \\
		
		\hline
			
	\end{tabular}}
	\label{tab:tab:Experiment1.3FourPatches5DegreeRotation}
\end{table*}

\begin{table*}[!ht]
	\caption{}
	\centering
	\resizebox{\textwidth}{!}{\begin{tabular}{|c|c|ccc|ccc|ccc|c|c|c|c|}
		\hline
		% Headers
		\multirow{2}{*}{\textbf{Runs}} & 
		\multirow{2}{*}{\textbf{VF}} & 
		\multicolumn{3}{c|}{\textbf{Cirrus}} & 
		\multicolumn{3}{c|}{\textbf{Spectralis}} & 
		\multicolumn{3}{c|}{\textbf{Topcon}} & 
		\multicolumn{1}{c|}{\multirow{2}{*}{\textbf{IRF}}} & 
		\multirow{2}{*}{\textbf{SRF}} & 
		\multirow{2}{*}{\textbf{PED}} & 
		\multirow{2}{*}{\textbf{Fluid}} \\ \cline{3-11} & &
		\multicolumn{1}{c}{\textbf{IRF}} & 
		\multicolumn{1}{c}{\textbf{SRF}} & 
		\textbf{\textbf{PED}} & 
		\multicolumn{1}{c}{\textbf{IRF}} & 
		\multicolumn{1}{c}{\textbf{SRF}} & 
		\textbf{PED} & 
		\textbf{IRF} & 
		\textbf{SRF} & 
		\textbf{PED} & 
		\multicolumn{1}{c|}{} & & & \\ 
			
		\hline
			
		\textbf{Run 32} & 1 & \multicolumn{1}{c|}{0.852} & \multicolumn{1}{c|}{0.918} & 0.934 & \multicolumn{1}{c|}{0.645} & \multicolumn{1}{c|}{0.821} & 0.852 & \multicolumn{1}{c|}{0.818} & \multicolumn{1}{c|}{0.929} & 0.755 & 0.799 & 0.905 & 0.841 & 0.742 \\
		
		\hline
			
	\end{tabular}}
	\label{tab:Experiment1FinalResults}
\end{table*}

\subsection{Experiment 2}

\section{Intermediate Slice Synthesis}\label{IntermediateSliceSynthesis}

\subsection{Experiment 3}

\subsection{Experiment 4}

\section{Fluid Volume Estimation}\label{FluidVolumeEstimation}

\subsection{Experiment 5}

\subsection{Experiment 6}