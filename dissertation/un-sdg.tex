\chapter*{UN Sustainable Development Goals} \label{chap:UnitedNations}

The United Nations Sustainable Development Goals (SDGs) provide a global framework to achieve a more equitable, healthy, and sustainable future for all. Comprising 17 different goals, this framework addresses pressing world's challenges, including poverty, hunger, health, and education.
\par
This dissertation contributes to these SDGs by advancing automated tools for the detection and characterization of retinal diseases, which are among the leading causes of vision impairment and blindness. The proposed methods support the ``SDG 3: Ensure healthy lives and promote well-being for all at all ages'', by developing fast, automatic, scalable, and accurate diagnostic tools that assist in the early identification and monitoring of retinal conditions. Since these solutions are fast, accessible, and easily deployed, they have the potential to be used in developing countries, where resources and access to trained specialists are scarce, supporting Target 3.d., which aims to support the capacity of all countries for early detection and monitoring of health risks. Additionally, through the prediction and monitoring of retinal diseases, this technology can also reduce the load on the healthcare systems, while improving decision-making and access to more equitable healthcare, once again addressing Target 3.d.
\par
In addition to the healthcare contributions, the studied technology can also be implemented as a educational material, providing an alternative tool for the understanding and visualization of retinal fluid pathologies, aligning with ``SDG 4: Ensure inclusive and equitable quality education and promote lifelong learning opportunities for all''. This addresses Target 4.4, which focuses on increasing of the number of young adults with relevant technical skills, and Target 3.c, which aims to improve the training of healthcare personnel.
\par
The relevant SGDs, their associated contributions, and performance metrics are explained in the following table:

\begin{description}
\item [SDG 3]
Ensure healthy lives and promote well-being for all at all ages
\item [SDG 4]
Ensure inclusive and equitable quality education and promote lifelong learning opportunities for all
\end{description}

\begin{center}
	\begin{tabular}{|l|l|p{58mm}|p{52mm}|}
		\hline
		\textbf{SGD} & \textbf{Target} & \textbf{Contribution} & \textbf{Performance Indicators and Metrics} \\ 
		\hline
		\hline
		\multirow{3}{*}{3} 
		& 3.d & Developed a GAN that generates intermediate slices between two known scans, increasing the spatial resolution and continuity of OCT volumes. & Reduction in misdiagnosis cases; Increased detection of early-stage retinal conditions.\\
		\cline{2-4}
		& 3.d & Designed an automatic fluid volume estimation pipeline that integrates retinal fluid segmentation and slice generation, and can accurately calculate the quantity of each fluid in type (IRF, SRF, and PED) an OCT volume, enabling disease progression monitoring and reducing clinician workload. & Reduction in clinician time spent per patient; Improved diagnostic consistency across clinicians.\\
		\cline{2-4}
		& 3.c & \parbox[t]{58mm}{Developed a segmentation model that performs accurate retinal fluid segmentation in OCT B-scans automatically, and can be used as an educational tool for understanding of retinal pathologies.} & \parbox[t]{52mm}{Improvement in diagnostic skills before and after using this tool; Increased acceptability of AI tools in the clinical workflow.}\\
		\cline{1-2}
		4 & 4.4 & & \\
		\hline
	\end{tabular}
\end{center}