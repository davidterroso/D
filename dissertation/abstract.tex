\chapter*{Resumo}
Danos causados na retina podem resultar em graves deficiências visuais, limitando  significativamente a qualidade de vida dos pacientes. Doenças como a degenerescência macular da idade (AMD), edema macular diabético (DME) e a oclusão da veia central da retina (RVO) estão entre as principais causas de lesão da retina e caracterizam-se normalmente pela presença e acumulação de fluído na retina.
\par
As características destes fluídos, visíveis em \textit{scans} de tomografia de coerência ótica (OCT), são importantes biomarcadores para a progressão de doenças. Contudo, a anotação manual destes fluídos é laboriosa, demorada e propensa a variabilidade entre observadores, o que motiva a necessidade de métodos de caracterização automáticos.
\par
Esta dissertação explora técnicas atuais de aprendizagem profunda para a segmentação semântica de fluídos retinais, visando a segmentação de fluído intra-retiniano (IRF), fluído sub-retiniano (SRF), e do descolamento do epitélio pigmentar (PED) em dois conjuntos de dados diferentes.
\par
Adicionalmente, o trabalho investiga o uso de modelos generativos, como redes adversariais generativas (GANs) para sintetizar \textit{scans} intermédios de OCT, com o objetivo de melhorar a resolução entre \textit{scans} num volume de OCT. Os resultados de segmentação no conjunto de dados com resolução melhorada são utilizados para calcular o volume de fluído nestes exames, um biomarcador importante na progressão das doenças na retina mencionadas anteriormente.
\par
Os resultados experimentais mostram que o modelo de segmentação alcançou resultados comparáveis aos da literatura, particularmente quando se utilizam \textit{patches} maiores que contribuem com contexto anatómico suficiente para a segmentação. Por outro lado, os modelos generativos, embora menos explorados em OCT, obtiveram resultados promissores na síntese de \textit{slices} intermédias realistas. Esta combinação de segmentação e geração permite uma caracterização mais robusta dos fluídos da retina, assegurando uma maior confiança nos volumes estimados.
\par
Apesar dos resultados obtidos serem encorajadores, existem ainda bastantes limitações, particularmente na generalização do modelo de segmentação para imagens externas e nos detalhes das imagens geradas e das máscaras de fluído previstas. Trabalhos futuros devem assim explorar arquiteturas avançadas, como mecanismos de atenção, e funções de penalização percetuais, com o objetivo de melhorar os detalhes e a robustez dos modelos em diversas condições de imagem.

\chapter*{Abstract}
The damage of the retina can result in significant visual impairment, severely limiting the quality of life of patients. Diseases such as age-related macular degeneration (AMD), diabetic macular edema (DME), and retinal vein occlusion (RVO) are among the leading causes of retinal damage and are typically characterized by the presence and accumulation of fluid in the retina. 
\par
The characteristics of these fluids, visible in retinal optical coherence tomography (OCT) scans, serve as important biomarkers for disease progression. However, the manual annotation of these fluids is laborious, time consuming, and prone to inter-observer variability, which motivates the need for automatic characterization approaches.
\par
This dissertation explores the state-of-the-art deep learning techniques for semantic segmentation of retinal fluids, targeting the segmentation of intra-retinal fluid (IRF), sub-retinal fluid (SRF), and pigment epithelial detachment (PED) in two different datasets. 
\par
Additionally, the work investigates the use of generative models, such as the generative adversarial networks (GANs), for synthesizing intermediate OCT slices, aiming to enhance the inter-slice resolution throughout an OCT volume. The segmentation results in the enhanced dataset are used in the for computing the fluid volume in the scans, an important biomarker in the progression of the aforementioned retinal diseases.
\par
The experimental results show that the segmentation model achieved comparable outcomes to existing approaches in the literature, particularly when using as input larger patches that contribute with enough anatomic context for the segmentation. Meanwhile, the generative models, though less explored in OCT, demonstrate promising results in the synthesis of realistic interpolated slices. This combination of segmentation and generation supports a more robust characterization of the retinal fluids, providing more confidence to the fluid volumes estimated.
\par
Despite the encouraging obtained results, several limitations remain, particularly in the segmentation model's generalization to external datasets and in the fine-detail realism of the generated slices and segmentation masks. Future works should explore advanced architectures, such as attention mechanism, and perceptual loss functions, to improve detail and robustness across diverse imaging conditions.