\chapter{Introduction}
The vision is the most important and complex sense in humans, playing a critical role in our orientation in the world \parencite{Hutmacher2019}. However, the health of the retina, an important part of the eye, can be compromised by multiple diseases that lead to fluid accumulation. The characterization of the fluid present in the retina is important to assess the progression of diseases such as age-related macular degeneration (AMD), diabetic macular edema (DME), and macular edema secondary to retinal vein occlusion (RVO) \parencite{Bogunovic2019a}.
\par
AMD affects the macular region of the retina, leading, in later stages, to a significant and permanent loss of central visual acuity, which has a severe impact on the patient's quality of life. In patients with AMD, the formation of new blood vessels can occur, which leak fluid, lipids, and blood into the retina, resulting in the formation of retinal fluid \parencite{Lim2012}. It is one of the leading causes of visual impairment with an expected effect on 300 million people by 2040 \parencite{Mitchell2018}.
\par
In patients with diabetes mellitus, DME represents the most common cause of visual impairment, affecting approximately 150 million people worldwide, as of 2015. It is anticipated that this number will increase as the prevalence of diabetes in developed countries is growing \parencite{Musat2015}. The fluid accumulation is caused by a disruption of the blood-retinal barrier, which allows fluid to accumulate in the intraretinal layers of the macula, resulting in retinal thickening (edema) \parencite{Bhagat2009, Bandello2019}.
\par
Affecting 16 million people worldwide, RVO represents a significant cause of vision loss in older individuals. The occlusion of the retinal vein can result in swelling of the optic disc, which leads to a reduction in visual acuity \parencite{Wong2010}. 
\par
The presence of intraretinal fluid (IRF) is a defining criterion of DME and RVO, and two in every three patients with AMD present this type of fluid. The majority of patients with AMD and 30\% of the patients with DME and RVO have subretinal fluid (SRF). Pigment epithelial detachments (PED) occur more frequently in patients with AMD \parencite{Bogunovic2019a}.
\par
Therefore, retinal fluids are important for the classification and progression of these diseases and can be observed through retinal optical coherence tomography (OCT) \parencite{Bogunovic2019a}. OCT is a non-invasive imaging technique that analyzes the light behavior (such as its reflection, absorption, and time-of-flight) to estimate the spatial dimensions of the tissue's structure \parencite{Huang1991}. This allows for \textit{in vivo} visualization of the individual retinal layers within the posterior segment of the eye. An OCT is composed of multiple consecutive cross-sectional 2D images that, when stacked, form a volumetric representation of the posterior segment. Each of these two-dimensional images is called a B-scan. In Figure \ref{fig:OCTVolume}, these concepts are illustrated, showing a B-scan and the three-dimensional representation of the posterior segment of the eye. The OCT resolution is sufficiently high to assess the tissue integrity, the retinal layers, and the fluids present \parencite{Drexler2008, Viedma2022}. There are multiple devices used for the acquisition of OCT volumes, resulting in different image attributes across the same technique, such as inter-slice distance, image quality, and appearance \parencite{Bogunovic2019a}. 

\begin{figure}[!ht]
	\centering
	\includegraphics[width=0.5\linewidth]{figures/OCTVolume.png}
	\caption{OCT B-scans (A), acquired at fixed intervals along the azimuthal axis, form a volume representation of the posterior segment of the eye (B) \parencite{Jain2010}.}
	\label{fig:OCTVolume}
\end{figure}

\par
The classification of the fluid is dependent on its location within the retina. There are three different categories: IRF, which is situated in the inner and outer layers of the retina; SRF, positioned between the outer nuclear layer (ONL) and the retinal pigment epithelium (RPE); and PED, which appear beneath the RPE \parencite{Bogunovic2019a}. Figure \ref{fig:SegmentedFluidsOCT} shows the characteristics and positions of these fluids on an OCT B-scan, and Figure \ref{fig:RetinalLayers} exhibits the retinal layers in the OCT scan of a healthy patient.
\par
\begin{figure}[!ht]
	\centering
	\includegraphics[width=0.75\linewidth]{figures/SegmentedFluidsOCT.png}
	\caption{The three distinct fluid types on an OCT B-scan: IRF in red, SRF in green, and PED in blue \parencite{Bogunovic2019a}.}
	\label{fig:SegmentedFluidsOCT}
\end{figure}
\begin{figure}[!ht]
	\centering
	\includegraphics[width=1\linewidth]{figures/RetinalLayers}
	\caption{OCT scan of the retinal layers \parencite{Almonte2020}.}
	\label{fig:RetinalLayers}
	\footnotesize
	\justifying
	ILM, internal limiting membrane; RNFL, retinal nerve fibre layer; GCL, ganglion cell layer; IPL, inner plexiform layer; INL, inner nuclear layer; OPL, outer plexiform layer; ELM, external limiting membrane; PR-IS/OS, photoreceptor inner segment/outer segment; RPE, retinal pigment epithelium; BM, Bruch's membrane. 
\end{figure}
\par
By segmenting the fluids detected in the B-scans, their volume can be estimated and used as a progression marker of the mentioned retinal diseases. However, manual segmentation is laborious, expensive, and prone to bias, which motivates the search for automatic methods \parencite{Viedma2022}. 
\par
In OCT imaging, the precision of the estimated volume is not only dependent on the quality of the segmentation, but also on the inter-slice distance \parencite{Lopez2023}. By reducing this distance and improving the OCT volume's resolution along this axis, the fluid segmentation becomes more accurate, when done both by clinicians and by segmentation models that leverage multiple consecutive slices as input \parencite{Selvi2013}. Given that the inter-slice space is reduced, the estimated segmented volume will also be closer to the real fluid volume.
\par
Considering the previous statements, the general objectives of this dissertation is to conduct an analysis of retinal OCT scans, classifying the retinal fluids in three distinct types (IRF, SRF, and PED) and quantifying their respective volumes. Another important objective is to increase the inter-slice resolution of the OCT volumes, with the aim of improving the fluid volume estimation. The specific objectives are as follows:  
\begin{enumerate}
	\item Develop different 2D deep learning models for multi-class segmentation of retinal fluids (IRF, SRF, and PED) in OCT volumes.
	\item Evaluate the performance of the best segmentation model and estimate the volume of each fluid using its predicted masks.
	\item Use a generative model for synthesizing intermediate slices in OCT volumes, generating one slice between two real slices in order to improve the inter-slice resolution of the volume, while assessing the quality of these generated images.
	\item Investigate the impact of intermediate slices synthesis on the fluid volume estimation by the segmentation models. 
\end{enumerate}
\par
Apart from the ``Introduction'', this dissertation is composed of the following chapters: ``Literature Review'', ``Methods and Materials'', ``Experiments'', ``Results and Discussion'', and ``Conclusion''. In the ``Literature Review'' chapter, an analysis is performed on the latest papers in the field of retinal fluid segmentation using 2D deep learning networks, as well as the latest publications on inter-slice resolution enhancement. The ``Methods and Materials'' chapter details the selection of the dataset for the methods applied in the experiments performed during the dissertation. The ``Experiments'' chapter presents how the methods introduced in the previous chapter were applied in practice, through experiments on fluid segmentation, intermediate slice synthesis, and fluid volume estimation. In the ``Results and Discussion'' chapter, the results from each experiment are shared, showing the performance of each model in its respective task. The differences in performance between experiments, the relationships with the implemented changes, and comparisons with results from the literature are also discussed. Finally, the ``Conclusion'' summarizes the main findings from the experiments performed and suggests directions for further research, while exposing some limitations of the study.
